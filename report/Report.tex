\documentclass[12pt]{ieeeconf}
\usepackage[margin=1in]{geometry}
\usepackage[english]{babel}
\usepackage[utf8x]{inputenc}
\usepackage{amsmath}
\usepackage{graphicx}
\usepackage{siunitx}
\usepackage{float}
\usepackage{url}
\usepackage{caption}
\usepackage{tocloft}
\usepackage{xcolor}
\usepackage{hyperref}
\usepackage{longtable}
\hypersetup{
  colorlinks = true,
  linkcolor = [rgb]{0,0,0.5},
  urlcolor = [rgb]{0,0,0.5}
}
\usepackage{lettrine}

\graphicspath{ {./images/} }
\renewcommand\cftaftertoctitle{\par\noindent\hrulefill\par\vskip-0.5em}

\newcommand\blfootnote[1]{%
  \begingroup
  \renewcommand\thefootnote{}\footnote{#1}%
  \addtocounter{footnote}{-1}%
  \endgroup
}

\title{Breast Cancer Cellularity Prediction from H\&E Images Challenge}
\author{Raj Patel, Blaze Kotsenburg}

\begin{document}
\begin{titlepage}
  \newcommand{\HRule}{\rule{\linewidth}{0.5mm}} % Defines a new command for the horizontal lines, change thickness here
  
  \vspace*{\fill}
  \center % Center everything on the page
  
  { \huge \bfseries Breast Cancer Cellularity Prediction from H\&E Images Challenge}\\
  \HRule \\[1cm]

  \large Raj Patel\\
  \large Blaze Kotsenburg\\[1.5cm]

  \normalsize ECE 6960 - Deep Learning for Image Analysis\\
  \normalsize \today\\[4cm]
  
  \vspace*{\fill}
\end{titlepage}

\maketitle
\begin{abstract}
Breast cancer affects about 1 in 8 women in the United States alone \cite{web1}. Being able to identify breast cancer in its early stages is of the utmost importance. Deep learning can be used to detect breast cancer from whole slide images of breast cancer hematoxylin and eosin stained pathological slides with high accuracies. This can save human labor and help detect breast cancer in its early stages, preventing it from spreading to neighboring cells. This research paper focuses on applying multiple convolutional neural network (CNNs) models to whole slide images of breast cancer to increase prediction accuracy of cancer cell detection.
\end{abstract}

\begin{keywords}
  Convolutional Neural Network (CNN), Deep Learning, ResNet
\end{keywords}

\blfootnote{This paper was submited for review on \today.}
\blfootnote{R. Patel is with the Department of Computer Engineering at the University of Utah, Salt Lake City, UT 84101 USA (e-mail: raj.patel@utah.edu).}
\blfootnote{B. Kotsenburg is with the Department of Computer Engineering at the University of Utah, Salt Lake City, UT 84101 USA (e-mail: bkotsenburg@gmail.com).}

\section{Introduction}
\lettrine{T}{he}

\section{Literature Survey}


\section{Materials and Methods}


\section{Experiments}


\section{Conclusion}


\nocite{*}
\bibliographystyle{IEEEtran}
\bibliography{IEEEabrv,bib/ref.bib}
\end{document}

