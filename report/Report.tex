\documentclass[11pt]{ieeeconf}
\usepackage[margin=1in]{geometry}
\usepackage[english]{babel}
\usepackage[utf8x]{inputenc}
\usepackage{amsmath}
\usepackage{graphicx}
\usepackage{siunitx}
\usepackage{float}
\usepackage{url}
\usepackage{caption}
\usepackage{tocloft}
\usepackage{xcolor}
\usepackage{hyperref}
\usepackage{longtable}
\hypersetup{
  colorlinks = true,
  linkcolor = [rgb]{0,0,0.5},
  urlcolor = [rgb]{0,0,0.5}
}
\usepackage{lettrine}

\graphicspath{ {./images/} }
\renewcommand\cftaftertoctitle{\par\noindent\hrulefill\par\vskip-0.5em}

\newcommand\blfootnote[1]{%
  \begingroup
  \renewcommand\thefootnote{}\footnote{#1}%
  \addtocounter{footnote}{-1}%
  \endgroup
}

\title{Breast Cancer Cellularity Prediction from H\&E Images Challenge}
\author{Raj Patel, Blaze Kotsenburg}

\begin{document}
\begin{titlepage}
  \newcommand{\HRule}{\rule{\linewidth}{0.5mm}} % Defines a new command for the horizontal lines, change thickness here
  
  \vspace*{\fill}
  \center % Center everything on the page
  
  { \huge \bfseries Breast Cancer Cellularity Prediction from H\&E Images Challenge}\\
  \HRule \\[1cm]

  \large Raj Patel\\
  \large Blaze Kotsenburg\\[1.5cm]

  \normalsize ECE 6960 - Deep Learning for Image Analysis\\
  \normalsize \today\\[4cm]
  
  \vspace*{\fill}
\end{titlepage}

\maketitle
\begin{abstract}
Breast cancer affects about 1 in 8 women in the United States alone \cite{web1}. Being able to identify breast cancer in its early stages is of the utmost importance. Deep learning can be used to detect breast cancer from whole slide images of breast cancer hematoxylin and eosin stained pathological slides with high accuracies. This can save human labor and help detect breast cancer in its early stages, preventing it from spreading to neighboring cells. This research paper focuses on applying multiple convolutional neural network (CNNs) models to whole slide images of breast cancer to increase prediction accuracy of cancer cell detection.
\end{abstract}

\begin{keywords}
  Convolutional Neural Network (CNN), Deep Learning, ResNet
\end{keywords}

\blfootnote{This paper was submited for review on \today.}
\blfootnote{R. Patel is with the Department of Computer Engineering at the University of Utah, Salt Lake City, UT 84101 USA (e-mail: raj.patel@utah.edu).}
\blfootnote{B. Kotsenburg is with the Department of Computer Engineering at the University of Utah, Salt Lake City, UT 84101 USA (e-mail: bkotsenburg@gmail.com).}

\section{Introduction}
\lettrine{B}{reast} cancer remains one the most commonly diagnosed cancers in women, apart from skin cancers \cite{carol}. It was estimated that in 2013 there would be approximately 232,340 new cases of invasive breast cancer and 39,620 breast cancer deaths in US women alone \cite{carol}. Recorded cases of breast cancer and mortality rates are expected to continue to increase in the future. Because of these statistics, breast cancer research remains a top priority for research in the biomedical field due to its prevalence in women.

Current methods used to detect breast cancer consist of breast exams, mammograms, breast ultrasounds, biopsy of breast cells, or magnetic resonance imaging (MRI) \cite{web2}. Breast exams require a doctor to examine lymph nodes near the armpit region to detect any abnormalities. Exams are generally the first step in the screening process. If any abnormalities are found, further screening will be needed. Mammograms take x-rays of the breast, producing a visualization of  any abnormalities that may be present in deeper tissue. Ultrasounds create a similar screening to x-rays. Ultrasounds can detect lumps and determine whether they are solid mass or a filled with fluid (cyst). Biopsies, which are probably the most promising test, take core tissue samples from suspicious areas. The tissue samples are sent off to a laboratory to be tested and examined by experts. Biopsies also allow for experts to determine the type of cell present in the tissue sample, which is beneficial for diagnosis. Finally, MRI’s are used to produce images from magnet and radiowaves. Images are then observed by a radiologist to determine if any anomalies exist in the breast \cite{web2}. 

\section{Literature Survey}


\section{Materials and Methods}


\section{Experiments}


\section{Conclusion}


\nocite{*}
\bibliographystyle{IEEEtran}
\bibliography{IEEEabrv,bib/ref.bib}
\end{document}

